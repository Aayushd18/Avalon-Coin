%%%% IACR Transactions TEMPLATE %%%%
% This file shows how to use the iacrtrans class to write a paper.
% Written by Gaetan Leurent gaetan.leurent@inria.fr (2020)
% Public Domain (CC0)


%%%% 1. DOCUMENTCLASS %%%%
\documentclass[journal=tosc,preprint]{iacrtrans}
%%%% NOTES:
% - Change "journal=tosc" to "journal=tches" if needed
% - Change "submission" to "final" for final version
% - Add "spthm" for LNCS-like theorems


%%%% 2. PACKAGES %%%%
\usepackage{times,amsmath,amsthm,amsfonts,eucal,graphicx,listings,booktabs,lipsum} % Example package -- can be removed


%%%% 3. AUTHOR, INSTITUTE %%%%
\author{Aayush Deshmukh\inst{1,2} \and Aman Khan\inst{1} \and Manohar Das\inst{1}}
\institute{
  Institute A, City, Country, \email{jane@institute}
  \and
  Institute B, City, Country, \email{john@institute}
}
%%%% NOTES:
% - We need a city name for indexation purpose, even if it is redundant
%   (eg: University of Atlantis, Atlantis, Atlantis)
% - \inst{} can be omitted if there is a single institute,
%   or exactly one institute per author


%%%% 4. TITLE %%%%
\title{PRINT Cipher}
%%%% NOTES:
% - If the title is too long, or includes special macro, please
%   provide a "running title" as optional argument: \title[Short]{Long}
% - You can provide an optional subtitle with \subtitle.

\begin{document}

\maketitle


%%%% 5. KEYWORDS %%%%
\keywords{Something \and Something else}


%%%% 6. ABSTRACT %%%%
\begin{abstract}
  In this paper we prove that the One-Time-Pad has perfect security.

\end{abstract}


%%%% 7. PAPER CONTENT %%%%
\section{Introduction}

Widely used primitives like the AES~\cite{AES} do not have perfect
security, and can be analysed with linear
cryptanalysis~\cite{EC:Matsui93}, differential
cryptanalysis~\cite{JC:BihSha91}, or differential power
analysis~\cite{C:KocJafJun99}.  We show that the One-Time-Pad is
unconditionally secure in \autoref{sec:main}.



\section{Main Result}
\label{sec:main}

\subsection{Sbox Analysis}

The sbox for the PRINT cipher is a 3-bit to 3-bit. Since input is 3-bit so for a b-bit block, the sbox is applied $\frac{b}{3}$ parallely. The current state for the sbox is a $\frac{b}{3}$ words, for each word same sbox is used and the next state is the concatenation of outputs.It is a balanced sbox and has a linear structure.The sbox is given in the following table :- \newline

\begin{table}[ht]
	\centering
	\resizebox{10cm}{!}{%
		\begin{tabular}{|l||l|l|l|l|l|l|l|l|}
			\hline
			\multicolumn{1}{|c||}{x} & \multicolumn{1}{c|}{0} & \multicolumn{1}{c|}{1} & \multicolumn{1}{c|}{2} & \multicolumn{1}{c|}{3} & \multicolumn{1}{c|}{4} & \multicolumn{1}{c|}{5} & \multicolumn{1}{c|}{6} & \multicolumn{1}{c|}{7} \\ \hline
			s{[}x{]}                & 0                      & 1                      & 3                      & 6                      & 7                      & 4                      & 5                      & 2                      \\ \hline
		\end{tabular}%
	}
\end{table}

\subsubsection{Difference Distribution Table}
The sbox has a differential branch number defined as min\textsubscript{v, w $\neq$v} \{wt(v $\oplus$ w) + wt(S(v) $\oplus$ S(w))\} of \textbf{2}. The difference distribution table (ddt) which is generated using Sage is as follows :- \newpage
\begin{table}[h]
	\centering
	\resizebox{8cm}{!}{%
		\begin{tabular}{@{}|l|llllllll|@{}}
			\toprule
			& \multicolumn{1}{c}{0} & \multicolumn{1}{c}{1} & \multicolumn{1}{c}{2} & \multicolumn{1}{c}{3} & \multicolumn{1}{c}{4} & \multicolumn{1}{c}{5} & \multicolumn{1}{c}{6} & \multicolumn{1}{c|}{7} \\ \midrule
			0 & 0                     & 0                     & 0                     & 0                     & 0                     & 0                     & 0                     & 0                      \\
			1 & 0                     & 2                     & 0                     & 2                     & 0                     & 2                     & 0                     & 2                      \\
			2 & 0                     & 0                     & 2                     & 2                     & 0                     & 0                     & 2                     & 2                      \\
			3 & 0                     & 2                     & 2                     & 0                     & 0                     & 2                     & 2                     & 0                      \\
			4 & 0                     & 0                     & 0                     & 0                     & 2                     & 2                     & 2                     & 2                      \\
			5 & 0                     & 2                     & 0                     & 2                     & 2                     & 0                     & 2                     & 0                      \\
			6 & 0                     & 0                     & 2                     & 2                     & 2                     & 2                     & 0                     & 0                      \\
			7 & 0                     & 2                     & 2                     & 0                     & 2                     & 0                     & 0                     & 2                      \\ \bottomrule
		\end{tabular}%
	}
\end{table}

\subsubsection{Linear Approximation Table}
The linear branch number which is defined as min\textsubscript{$\alpha$ $\neq$, $\beta$, LAM($\alpha$,$\beta$)$\neq$0}\{wt($\alpha$) + wt($\beta$)\} for this sbox is \textbf{2}. The linearity of this sbox is \textbf{4}. The linear approximation table generated from Sage is as follows:-
% Please add the following required packages to your document preamble:
% \usepackage{graphicx}
\begin{table}[h]
	\centering
	\resizebox{8cm}{!}{%
		\begin{tabular}{|c|cccccccc|}
			\hline
			& 0 & 1  & 2  & 3  & 4  & 5  & 6  & 7  \\ \hline
			0 & 4 & 0  & 0  & 0  & 0  & 0  & 0  & 0  \\
			1 & 0 & -2 & 0  & 2  & 0  & 2  & 0  & 2  \\
			2 & 0 & 0  & 2  & 2  & 0  & 0  & 2  & -2 \\
			3 & 0 & 2  & -2 & 0  & 0  & 2  & 2  & 0  \\
			4 & 0 & 0  & 0  & 0  & 2  & -2 & 2  & 2  \\
			5 & 0 & 2  & 0  & 2  & 2  & 0  & -2 & 0  \\
			6 & 0 & 0  & 2  & -2 & 2  & 2  & 0  & 0  \\
			7 & 0 & 2  & 2  & 0  & -2 & 0  & 0  & 2  \\ \hline
		\end{tabular}%
	}
\end{table}

\subsubsection{Additional Properties of Sbox}
\textbf{1.} The component funcion in 3 variables in algebraic normal form of the sbox is
\begin{center}
	\textbf{x0*x2 + x0 + x1*x2}
\end{center} 

\noindent\textbf{2.} The interpolation polynomial for the sbox is
\begin{center}
	\textbf{(a + 1)x\textsuperscript{6} + (a\textsuperscript{2} + a + 1)x\textsuperscript{5} + (a\textsuperscript{2} + 1)x\textsuperscript{3}}
\end{center}


\noindent\textbf{3. } The polynomials which satisfy the sbox is
\begin{itemize}
	\item x0*x2 + x0 + x1 + y1
	\item x0*x1 + x0 + x1 + x2 + y2
	\item x0*y1 + x0 + x2+ y1 + y2
	\item x0*y2 + x1 + y1
	\item x1*x2 + x0 + y0
	\item x1*y0 + x1 + x2 + y0 + y2
	\item x0*y0 + x1*y1 + x2 + y2
	\item x1*y2 + x0 + x1 + y0
	\item x2*y0 + x1 + y0 + y1
	\item x2*y1 + x0 + y0
	\item x0*y0 + x2*y2 + x0 + x1 + x2 + y0 + y1
	\item y0*y1 + x2 + y0 + y1 + y2
	\item y0*y2 + x1 + y1
	\item y1*y2 + x0+ y0 + y1
\end{itemize}
\hspace{0.5cm} x - input variables y - output variables \newline

\noindent\textbf{4.} Maximum degree of component function - 2\newline

\noindent\textbf{5.} Minimum degree of component function - 2\newline

\noindent\textbf{6.} Maximal differential probability - 0.25\newline

\noindent\textbf{7.} Absolute maximal linear bias - 2\newline

\noindent\textbf{8.} Relative maximal linear bias - 0.25\newline
%%%% 8. BILBIOGRAPHY %%%%
\bibliographystyle{alpha}
\bibliography{}
%%%% NOTES
% - Download abbrev3.bib and crypto.bib from https://cryptobib.di.ens.fr/
% - Use bilbio.bib for additional references not in the cryptobib database.
%   If possible, take them from DBLP.

\end{document}
